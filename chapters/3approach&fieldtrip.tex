%!TEX root = ../thesis.tex

\chapter{Data}
\label{ch:data}
%\section{Approach}
%Since there has not been done many research projects about \acrshort{shs} in Albania, this thesis will use a similar approach to other studies. The most common from Sub-Saharan Africa being the socioeconomic aspect, we will apply this evaluation. We will look at the usage of these systems and compare that to their capability. Their energy savings will be compared towards their cost. 
\section{Qualitative research}
We use a qualitative method for the analysis of consumption and general usage of the systems. To acquire this information we had a field trip with interviews of the projects participants. The field trip consisted of two days to gather our field data. Since we anticipated fewer than 30 interviews, the of data should be mostly qualitative. This journal article about qualitative interviews \citep{malterudSampleSizeQualitative2016} explains how we can increase our "information power" when narrowing down the research area. Giving more credibility to the data when it has less possibility for errors. Meaning we should try to find as many participants in the same situation as possible. The systems where donated to various organizations and households, but we focused on poor families living in rural areas.

Narrowing down to mostly poor rural households gave us data that would be more similar than data from randomly chosen participants. Poor families often live with some of the same issues and problem. They often have a restricted household economy, and tend to be aware of every cost they have. As poor families living in rural areas were also the largest group, they were the natural choice for the study. Two interviews were done with organizations that had several systems, were one was a social house for disabled elderly people - and the other was an eco-social farm. Among these, only the consumption data of the social house was deemed relevant for the study. As both would be anomalies in household compositions and economics. The social house consumption data was deemed relevant as it had the same purpose for usage as households.


\section{Simulation and Load Profile}
The \acrshort{shs} don't record any production in the system, meaning there is no way to extract the data from the system. Knowing the size of the PV panel and the battery size, we can use simulation to estimate this. With this production data, we can see the differences in seasons and analyze if the systems will last with it's projected use every day. 
To analyze the electricity usage of the systems, we create a load profile. This gives some quantifiable data to help analyze the usage of the system. Asking the specific consumption questions in Appendix \ref{apx:ftquestions}, we can create a profile of daily usage to estimate the consumption. Combined this will give us how much the systems are utilized based on their capability and actual usage. Simulink was used for all simulations that were not gathered from \acrshort{pvgis}.
%\subsection{Metrics}
%To analyze the socioeconomic value of the systems, we use metrics that are common in economical analysis. LCOE is common in energy economics and gives us a good base to work from. These systems are generally not known for being cheap for users to buy, and therefore they need to have a sufficient energy yield. To see if any household should invest in this we will use payback time. SHS is not a common product outside of off-grid areas, and is therefore not easily compared with payback time. As this area is grid connected, the payback can directly be compared to the cost of consumption from the grid. How utilized the system is, will be needed for all of these metrics. The utilization rate is also relevant to look at as it's own metric. If the utilization rate is low, these systems may have been better aid somewhere else. 

%To analyze the social benefits, we compare answers and discuss the effects. The daily load profile together with the simulated production will be used as the baseline for most of the economical evaluation.

%\subsection{Discussion}
%For the less tangible evidence from the interviews, we will discuss this in the results. Here we will look at what factors that are important to consider besides the metrics. 


\section{Field trip}
During the start of April 2025, the field trip took place in Shköder, Albania. During the field trip, "The Door Albania" organized the interviews with the participants of the project. The interviews were translated by representatives from "The Door Albania" between Albanian and English. Some data is gathered from observations, although most of it is from recorded interviews. All participants signed a written agreement to let us use their data in this study, and all data was anonymized to preserve their data privacy right. The study was also approved by Sikt, the Norwegian Agency for Shared Services in Education and Research.

\subsection{Observations}
Which type of SHS they had was observed and noted down for each interview. The angle and azimuth was also estimated with pictures and notes from the household visits. Observations of where the lights were placed, and how many lights were mounted was also noted with pictures. General observations for use in the discussion includes the state of the household, and their attitude towards the system. 

\subsection{Interviews}
There were conducted 14 interviews encompassing 20 SHS in total. Most of the interviews where one-to-one, each of them owning one system. Some participants had multiple systems, and some interviews where done with multiple participants. Households consisted of 14 of the total systems on 11 interviews, where one interview was done with three households together. Off-gird sheds was two interviews with two total systems. One social house interview being the remaining four systems. Poor households tended to answer a lot of the same answers, resulting in neatly gathered data.
