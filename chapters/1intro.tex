%!TEX root = ../thesis.tex

\chapter{Introduction}
\label{ch:intro}
\acrfull{shs} have been used for decades as humanitarian aid. They give light and utility to areas that are otherwise without access to electricity. Harvesting solar power, they provide renewable energy to off-grid areas in a household size system. Commonly used in areas with high irradiation and clear weather - optimal conditions for solar power systems. Less common is the use of theses systems in countries with high grid coverage. From 2000 until 2022, the number of people living without access to electricity has halved \citep{owid-energy-access}. Higher grid connection reduces the use of SHS as humanitarian aid.

This thesis takes a look at the use of SHS as economic and social aid in a developing country with high grid connection. The thesis uses Albania as a case study, which achieved full grid coverage during the last 20 years \citep{internationalrenewableenergyagencyTrackingSDG7Energy2023}. Albania is a sun rich middle-income country in the Adriatic sea, with a high rural population. 

Norwegian Aid and Bright Products cooperated in early 2024 with the distribution of 144 systems to Albania through the organization The Door Albania. The data for this thesis was gathered in a field trip on the spring of 2025, giving the participants one year of use with the system. The study focuses on how SHS function as humanitarian aid for the receivers, both economically and socially. Simulations of power generation and consumption are the basis of the economical analysis, using data gathered from interviews to make a daily load profile. The social part is directly founded on the answers from interviews and the noted observations made during the field trip. Present on the field trip was Bright Products, The Door Albania and \acrfull{iug}. 

Bright Products supplied the system specific information needed for their products, and has assisted with defining the problem statement. \acrshort{iug} has helped with defining the interview questions, and logging down observations. The thesis uses tools like \acrfull{pvgis}, Simulink and Matlab to simulate production data from the systems. \acrfull{sikt} has provided the necessary routines to ensure data privacy for the interview participants. 


\section{Objectives}
The objectives of this thesis is:
\begin{itemize}
\item Help humanitarian aid organizations provide the best aid they can, given their means.
\item Evaluate the use of SHS in a grid-connected area in a socioeconomic view.
\item Showcase the "Use the Sun" project and the results that the aid gave. 
\item Discuss the benefits of the project in relation to other humanitarian aid.
\end{itemize}
Evaluating the use of theses system in our case study has it's main objective in making humanitarian aid more valuable. The field of SHS lacks studies where the area has high grid coverage. Most humanitarian aid has a socioeconomic goal, wanting to improve the general living conditions of the recipients. An early follow-up of "Use the Sun" would show how this project could be expanded if proved beneficial. Discussing the results will give context to the data to provide the reader with more depth than objective data. 

\section{Approach and outline}
All data that is shown in results is gathered in interviews from the field trip. Supplementary data is presented before, and only referred to in the results. The approach to this thesis was to look at metrics for economical benefits based on empirical energy data, and analyze the social benefits based on tailored questions and observations. Using a daily load profile with consumption and production showed the actual usage of the systems. Creating structured data of the social benefits helped analyze the effect objectively. 

\textbf{Outline:}
\begin{itemize}
\item Background will explain the situation around SHS and Albania, giving context to the thesis.
\item Field trip will show details around the organization of the interviews and how they were conducted.
\item Theory will explain formulas, information and factors used to treat the data.
\item Method gives specific insights into how the data was treated, and what a researcher has to do to replicate the process. 
\item Results analyzes the data created from interviews.
\item Discussion adds depth to the data from results.
\end{itemize}







