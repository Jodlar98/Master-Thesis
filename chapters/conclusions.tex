%!TEX root = ../thesis.tex

\chapter{Conclusions}
\label{ch:conclusion}

The findings of this paper suggest that the benefits of SHS in developing middle-income countries with high grid coverage are mostly social, as opposed to economical. With payback time of over 50 years and LCOE of 19.7-26.5 EUR/kWh, the economical benefits of a SHS are negligible compared to the cost. SHS are fully utilized, consuming all of the electricity that they produce in winter time and generating a surplus in the summer time. Users show high interest in the systems, taking good care of the system and treating it as valuable. People particularly valued their systems as a backup in case of power outages, which were frequent in Albania. Some were illegally connected due to not being able to pay the electricity bill, and risked being disconnected at any time. Even with full grid coverage, there are still off-grid uses for a SHS - including sheds and households disconnected from the grid. Common alternative sources of energy such as gas and wood were used for cooking, hot water and heat - making the SHS fill the remaining need for lighting and charging electronics during power outages. Using a local organization for managing a SHS project was successful, giving participants a contact for maintenance and training. 

A potential area of further research is to further investigate the social benefits of having a solar powered backup system for households in countries with low grid stability. 