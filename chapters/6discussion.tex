%!TEX root = ../thesis.tex

\chapter{Discussion}
\label{ch:discussion}
This chapter aims to represent the data in context with the field trip. Discussing the results given i the previous chapter, this chapter refers to results as it is read fully. Uncertainty of data is cited, and future changes to the study. 
\section{Power simulations}
Tilt angle and azimuth was based on observations. Though fairly accurately described and taken pictures of, the angle is still estimated and individually grouped. No angle measuring instrument was used, and only a simple mobile compass was used to determine the azimuth. As the results show, the tilt angle and azimuth proved to give a significant reduced effectiveness of the panels. Still, not every panel could be placed in an optimal setting. Mounting the system firmly and safely was a greater priority for the installers that helped than the angle and azimuth was. Some of the households did not have the possibility of placing the panel optimally. Shading of the panels were not accounted for, but were generally not common among the participants.

Cleaning was a factor that the receivers was not informed about. In general they got the training they needed for the system, but maintaining the panels by washing was unknown by the local aid organization. All participants got this information during the interviews, and the organization informed other recipients of washing.

Seasonal differences in production and consumption is a simplification of the process that causes inaccuracy. As we did not have data besides what the participants consumed in winter and summer time, using two rigid dataset for consumption on a dynamic production dataset would also cause inaccuracy. Creating two distinct simulations also reduced the inconsistency of weather, as the dataset for irradiance was for 2023. Future studies could use integrated logging technologies to analyze the power data more precisely. SoH for batteries was not measured, as we had to prioritize tasks in our short time. We disregarded SoH estimation as the age would make the deterioration negligible. 

A general loss factor of 14\% could have been unjustified, and to analyze this actual measurements could have been done. As the system is using voltage conversion mechanisms for charging and discharging, the system could be more or less efficient than specified. The simulation should have used the median efficiency loss from azimuth and tilt angle, as only some panels were badly placed. The median reduction was 14\% and could have better represented the normal placement of panels. A soiling loss factor of 10\% could have been more specified based on each panel, but time budget of interviews did not give us the opportunity to inspect the panels closely.

Summer and winter data mostly showed how utilized the systems were. When simulating the combined data against a BH 800 system on figure \ref{result:fig:Winter_40wpp_all_losses}, the system showed complete usage of the produced power. Since the bottleneck is the winter time, this is a optimally sized system for it's use. It should be noted that the general simulations on the BH 800 were made with combing all systems consumption data, which included the data BH 600 and BH 300 systems. Since the sample size was low for each system, this way of combining the datasets could give a more accurate representation of consumption. On the system specific simulations, datasets were not as general as the combined. Simulations on figures \ref{result:fig:300_winter_soc}, \ref{result:fig:600_winter_soc} and \ref{result:fig:800_winter_soc} showed how each system was fully utilized in the winter. As the participants did not state that the battery would empty each day, there is a mismatch between the data. Most likely the consumption data is overestimated. Causes for the mismatch could be inaccurate descriptions in interviews, or inaccurate assumptions done during the structuring of the interview data.  


% Du kom hertil
\section{Economic}
Although section \ref{ch:back:sec:solar} refers to an LCOE of 0.044 USD per KWh, this does not directly apply to a SHS. The system includes lights, battery and is customized to a user friendly size. These functionalities are not included in a utility-scale PV system. Lower LCOE in utility-scale PV systems point towards a trend of declining costs of solar technology, which will benefit SHS in the future. As table \ref{tab:lcoe_qty_cons_5_condensed} with 5\% discount and consumption data shows, the LCOE is still around 20 EUR/kWh for the stated lifetime. Cost of electricity from the grid is 0.12 EUR/kWh, giving SHS 100 higher cost per unit of energy. This is partially from the low cost of electricity in Albania, being half of what the EU-27 average is. Rising electricity price could increase the profitability against the LCOE and payback time, but there was no real basis to predict this on a 6-7 years lifespan.  If one were to do a more fair analysis of the cost, the benefits of the utility should be considered and removed from the investment cost. The systems includes lighting sources, which would otherwise have had to be bought. For off-grid households or buildings, this can be a costly process that makes a SHS be the most affordable solution. Although the systems are marketed to a 6-7 years life cycle, it is likely that they will last longer. SHS are typically estimated to live for longer, being based on lifetime of battery. As section \ref{chap:method:sec:LCOE} explain, the lifetime of a battery can be up to 9000 cycles or about 24-25 years. This would reduce the LCOE to about 7.5 EUR/kWh when using consumption data and 5\% discount rate. 

Users would still give a high price for the system even though the pure economical payback time would be higher than a worthwhile investment. From the payback time in figure \ref{result:fig:paybacktime}, they would still have minimum of 25 years for a economical payback in terms of energy bill reduction. This shows that there are benefits beside the economical that the participants value. 

Overcharging was reported by almost all participants. \citep{likmeta2007} reports this as a problem as far back as 2007. The problem seems to be a know issue, and likely due to missing measurements of electricity consumption. When overcharging is that normal, Albanians may feel unease by being connected to a grid without control of their costs. Suddenly being charged five times higher than normal for power could break a fragile household economy. \citep{Hejsek2011} reports that some people where using power lavishly without seeing a change in the power bill, as the bill was estimated anyway. Even though we were there to interview about the SHS, participants expressed their frustration of the power system in response. Participants expressed a desire to be self-sufficient as solution against the overcharging from the electricity companies. Most then referred to a bigger SHS as the step towards this, although none had the means to invest in this.

\section{Social}
Participants reported that the system gave them freedom to use their own electricity, not reliant on the grid. This could have been because of the common power outages that often lasted a while, giving them a sense of control that they lacked without the system. Several reported feeling an increased safety with the system, giving them an emergency solution in a power outage. Illegally connected households particularly reported feeling safer with the system, knowing they had a backup system if they were to be disconnected from the grid. As gas and wood was a common energy source for cooking and heating water, electricity was mostly used for lighting, TV and washing clothes. The SHS would then be able to cover the need of lighting and phone charging, making the situation less critical than it would be without it.

One interviewed household was off-grid, where there was seven people in the household. They used only the SHS as light in addition to rechargeable electric lights where the lamps could not reach. Rechargeable lights were being charged by using the USB-A port on the SHS. Additionally, they charged a power bank to serve as extended battery capacity for the SHS. The household had grid infrastructure, meaning they choose not to connect to the grid. The two off-grid sheds had grid connection inside the household, but had not extended this into the shed. Participants with sheds explained that they would otherwise not have had electricity in the building, as the cost of extending the power was too costly. 

Power outages were common among the participants. Being frequent and lasting for a while, participants reported this being a source of worry. \citep{giz_kfw_2013}, a german organization working in Albania - took the measure of installing a EUR 20,000 system to elude the issue of power outages in the capital. Some rural areas were exposed to longer power outages than urban. Some reporting that there could be a day before they would get power again. Specially households with disabled and elderly valued that the system was independent of the grid. Having lights available in power outages made care easier. Participants reported using the light as a safety during the night, but had not used this previously in fear of high electricity costs. The social house reported that this allowed them to have meals in a power outage, where they previously had to cancel. Using rechargeable lights as their backup, this was not enough light to safely arrange meals. The SHS also served as an important safety for both the patients and the caretakers, as there was no emergency generator for the building.

Candles were the most common alternative to lighting during power outages. The use of candles is as mention in section \ref{ch:metod:powerout} connected to fire hazards. Participants were aware of the risk, and often cited removing the use of candles as one of the best use cases of the system. As removing candles was not an original question we had before the field trip, the participants were actually the ones that consistently brought this up. \citep{obengSolarPhotovoltaicElectrification2008} in Ghana cites PV systems as reducing the indoor smoke from kerosene lamps, and replaces the need for candles. Although the literature is referring to off-grid areas, frequent power outages create the need for alternative lighting sources like candles and kerosene lamps. \citep{Sinoruka2023} tells the story of a village in rural Albania that previously did not have street lamps, using torches for lights at night. Now they have gotten a PV system for their streetlamps, giving them light throughout the night.

Households reported a higher electricity payment reduction in figure \ref{res:fig:HouseholdElectricityEconomy} than the system would generate in terms of power. With the lowest reported saving of EUR 10, this is still above the optimally generated power price from any system. With three data samples, it is too uncertain to claim cost saving. Participants did seem to expect an energy saving from the system though, as we did not ask the participants of energy savings from the system. In fear of it being a leading question, we instead asked about consumption and energy costs. Participants were particularly aware of their energy costs, and took care to save energy where they could. 


\section{Organizational}
Local aid organization The Door Albania had selected the receivers of the SHS. From their work they had a network of receivers of aid, which they used select the participants of the project. In addition, they had dialogue with the municipality of Shköder to find more participants. The organization had a detailed list of the participants of the project, and served as a contact for the system. Participants knew that if there was an issue with the system, or if they needed training - the organization would help them. Participants often received additional help from the organization besides the SHS, and had regular communication. 

This way of structuring the aid seemed to be successful. \citep{chaureyAssessmentEvaluationPV2010} cites that the attitude of the user determines the success of the program, and the significance of local involvement in programs. Having this project driven by an established local aid organization could make participants feel more responsibility for the systems. During interviews, participants exclaimed their gratefulness for the systems and often thanked the organization. 

\section{Shortcomings}
The general shortcoming of this thesis is the number of variables gathered for the social evaluation. There is no value in gathering more data for the economical analysis, as this can be concluded with the data we have gotten. The data on SHS outside of off-grid areas is low, due to few projects being done in this area. Narrowing down on the social effects that the systems give would be the way forward. The study could have more data points within an even more narrow category. Some interviews were done in a social housing complex, where the municipality had put up modular units for poor families. This place had over 20 families with systems of which we only got to interview three in our time there. Narrowing down the sample group to only these would give more precise data. Nevertheless, the data we got should accurately represent the entire selection. As there were 187 systems given with 14 of them being reserves, we interviewed for over 1/10 of the systems. 

This study does not include data from other developing middle-income countries. Comparing the results of this study with other projects in developing countries with high grid coverage would be the next step for research in this field. Particular points of interest would be organizational, social, cultural and grid infrastructure. 
